
% -----------------------------------
% paper # 89 Klaus
% -----------------------------------

%\ \hline \ % to be removed before submission

Haxthausen and Peleska
\cite{isola-2016-haxthausen}
({\em On the Feasibility of a Unified Modelling and
Programming Paradigm})
argue that we should not expect there to be a single ``best'' 
unified modeling, programming and verification paradigm in the 
future. They, on the contrary, argue that a multi-formalism 
approach is more realistic and useful. Amongst the reasons 
mentioned is that multiple stake holders will not be able to agree 
on a formalism. The multi-formalism approach
requires to translate verification artifacts between different 
representations. It is illustrated
by means of a case study from the railway domain, how this can be
achieved, using concepts from the theory of institutions, formalized in category theory. 



