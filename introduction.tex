
\section{Motivation and Goals}
\label{sec:introduction}

Since the 1960s we have seen tremendous amount of scientific and 
methodological work in the fields of specification, design, and 
programming languages. In spite of the very high value of this work, however, this effort has found its limitation by the fact that we do not have a sufficient integration of these languages, as well as tools that support the development engineer in applying the corresponding methods and techniques. A tighter integration between specification and verification logics, graphical modeling notations, and programming languages is needed.

In a (possibly over) simplified view, as an attempt to impose some 
structure on this work, we can distinguish between three lines of 
work: formal methods, model-based engineering, and 
programming languages. Formal methods include, usually textual, 
formalisms such as VDM, CIP, Z, B, Event-B, ASM, TLA+, Alloy, and RAISE, but also more or less 
automated theorem proving systems such as Coq, Isabelle, and 
PVS. Such formalisms are usually based on mathematical 
concepts, such as functions, relations, set theory, etc. A 
specification typically 
consists of a signature, i.e. a collection of names and their types, and 
axioms over  the signature, constraining the values that 
the names can denote. A specification as such denotes a set of 
models, each providing a binding of values to the names, satisfying 
the axioms. Such formal methods usually come equipped with proof 
systems, such that one can prove properties of the specifications, 
for example consistency of axioms, or that certain theorems are 
consequences of the axioms. A common characteristic of 
these formalisms is their representation as text, defined by context-free grammars, and their formalization in terms of semantics and/or logical proof systems. 
%
In parallel one has seen several model checkers appearing, such as
SPIN, SMV, FDR, and UPPAAL. These usually prioritize efficient verification algorithms over expressive and convenient 
specification languages. Exceptions are more recent model checkers for programming languages, including for example Java PathFinder (JPF).

Starting later in the 1980s, the model-based 
engineering community developed graphical formalisms, most 
prominently represented by UML and later SysML. These formalisms 
offer graphical notation for defining data structures as ``nodes 
and edge'' diagrams, and behavioral diagrams such as state machines 
and message sequence diagrams. These formalisms specifically 
address the ease of adoption amongst engineers. It is clear that
these techniques have become more popular in industry than formal 
methods, in part likely due to the graphical nature of 
these languages. However, these formalisms are complex (the 
standard defining UML is much larger than the definition of any 
formal method or programming language), are incomplete (the UML
standard for example has no expression-language, although OCL is
a recommended add-on), and they lack commonly agreed up semantics. 
%%%
This is not too surprising as UML has been designed on the basis of an 
intuitive understanding of the semantics of its individual parts and concepts, 
and not under the perspective of a potential formal semantics ideally covering 
the entire UML. This leaves users some freedom of interpretation, in particular 
concerning the conceptual interplay of individual model types and often leads 
to misunderstandings, but it has still been sufficient in practice in order to 
support tool-based system development, even by providing, e.g., partial code 
generation. On the other hand, it is also responsible for the only very partial 
successes of the decades of attempts to provide formal semantics to UML. One may, 
therefore, argue that (the abstract syntax and intuitive semantics of) UML, as it 
stands, is not adequately designed to support a foundation in terms of a formal 
semantics. It would therefore be interesting to reconsider the design of UML 
with the dedicated goal to provide a formal semantics and thereby reach a next 
level of maturity.
%%%

Finally, programming languages have evolved over time, starting 
with numerical machine code, then assembly languages, and 
transitioning to higher-level languages with FORTRAN in the late 
1950s. Numerous programming languages have been developed since 
then. The C programming language has  since its creation in the 
early 1970s conquered the embedded software world in an 
impressive manner. Later efforts, however, have attempted to
create even higher-level languages. These include language such as
Java and Python, in which collections such as sets, lists and maps 
are built-in, either as constructs or as systems libraries. 
Especially the academic community has experimented with functional 
programming languages, such as ML, OCaml, and Haskell, and more 
recently the integration of object-oriented programming and 
functional programming, as in for example Scala.

Each of the formalisms mentioned above have advantageous features not owned by other formalisms. However, what is perhaps more important is that these formalisms have many language constructs in common, and to  such an extent that one can ask the controversial question: 
{\em Should we strive towards a unified view of modeling and 
programming?} 
%Less controversial questions are: 
%{\em What kind of infusion can be imagined in between formalisms?} 
%(what can each field learn from other fields),
%or: {\em how do we combine models written in different 
%languages?}. For example, can programming languages be extended to 
%support modeling, and can visualization become a standard part of 
%programming languages? Could we imagine one universal 
%specification and programming formalism? Should we accept that 
%models will be written in numerous languages, even within one 
%project, and instead focus on how to integrate models in different 
%languages? These are some of the questions that this track 
%addresses. 
It is the goal of the meeting to discuss the 
relationship between modeling and programming, with the possible 
objective of achieving an agreement of what a unification of these 
concepts would mean at an abstract level, and what it would bring 
as benefits on the practical level. What are the trends in the 
three domains: formal specification and verification,
model-based engineering, and
programming, which can be considered 
to support a unification of these domains. We want to discuss whether the time 
is ripe for another attempt to bring things closer together. 

