
\section{Motivation and Goals}

Since the 60s we have seen tremendous amount of scientific and methodological work in the fields of design and specification languages, programming languages, and modeling formalisms and concepts. In spite of the very high value of this work, however, this effort has found its limitation by the fact that we do not have a sufficient integration of programming, specification, and modeling languages as well as tools that support the development engineer in applying methods and techniques from all of these.

A closer look shows that a tight integration between specification and verification logic, programming languages, and graphical modeling notations as well as modeling concepts is needed. An additional difficulty is due to the fact that even in each of the fields of programming and programming languages, specification and verification logics, and graphical modeling notations and concepts, a rich variety of different approaches exist. Scientifically, this variety is highly necessary. For finding its way into engineering practice, however, the variety and the missing integration of the methods is a serious obstacle. It is the goal of the meeting to discuss the relationship between modeling and programming, with the possible objective of achieving an agreement of what a unification of these concepts would mean at an abstract level, and what it would bring as benefits on the practical level. What are the trends in the three domains: programming, formal specification and verification, and model-based engineering, which can be considered to be unifying these domains. We want to discuss whether the time is ripe for another attempt to bring things closer together.

The track is divided into four sessions: meta-modeling, model-based engineering (graphical tools), formal specification languages, and finally programming languages.

\kh{This section needs to be filled out.}
