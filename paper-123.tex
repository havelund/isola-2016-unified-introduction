
% -----------------------------------
% paper # 123 Bernhard
% -----------------------------------

%\ \hline \ % to be removed before submission

Naujokat, Neubauer, Margaria, and Steffen 
\cite{isola-2016-steffen} ({\em Meta-Level Reuse for Mastering 
Domain Specialization}) reflect on the distinction between modeling 
and programming in terms of {\textsc{what}}{} and {\textsc{how}}{}, 
and emphasize the importance of perspectives: what is a model (a 
{\textsc{what}}{}) for the one, may well be a program (a 
{\textsc{how}}{}) for the other. In fact, attempts to pinpoint 
technical criteria like executability or abstraction for clearly 
separating modeling from programming seem not to survive modern 
technical developments. Rather, the underlying conceptual cores 
continuously converge. 
% ---
What remains is the distinction of 
{\textsc{what}}{} and {\textsc{how}}{}, separating true purpose 
from its realization, i.e. providing the possibility of 
formulating the primary intent without being forced to 
over-specify. 
% ---
The paper 
argues that no unified general-purpose language can adequately 
support this distinction in general, and propose a meta-level 
framework for mastering the wealth of required domain-specific 
languages.